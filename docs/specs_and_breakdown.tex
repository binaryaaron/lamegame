\documentclass[titlepage]{article}\usepackage[]{graphicx}\usepackage[]{color}
%% maxwidth is the original width if it is less than linewidth
%% otherwise use linewidth (to make sure the graphics do not exceed the margin)
\makeatletter
\def\maxwidth{ %
  \ifdim\Gin@nat@width>\linewidth
  \linewidth
  \else
  \Gin@nat@width
  \fi
}
\makeatother


\usepackage{listings}
\definecolor{mygreen}{rgb}{0,0.6,0}
\definecolor{mygray}{rgb}{0.5,0.5,0.5}
\definecolor{mymauve}{rgb}{0.58,0,0.82}
\lstset{ %
  backgroundcolor=\color{white},   % choose the background color; you must add \usepackage{color} or \usepackage{xcolor}
  basicstyle=\footnotesize,        % the size of the fonts that are used for the code
  breakatwhitespace=false,         % sets if automatic breaks should only happen at whitespace
  breaklines=true,                 % sets automatic line breaking
  captionpos=b,                    % sets the caption-position to bottom
  commentstyle=\color{mygreen},    % comment style
  deletekeywords={...},            % if you want to delete keywords from the given language
  escapeinside={\%*}{*)},          % if you want to add LaTeX within your code
  extendedchars=true,              % lets you use non-ASCII characters; for 8-bits encodings only, does not work with UTF-8
  frame=single,                    % adds a frame around the code
  keepspaces=true,                 % keeps spaces in text, useful for keeping indentation of code (possibly needs columns=flexible)
  keywordstyle=\color{blue},       % keyword style
  language=Python,                 % the language of the code
  morekeywords={*,...},            % if you want to add more keywords to the set
  numbers=left,                    % where to put the line-numbers; possible values are (none, left, right)
  numbersep=5pt,                   % how far the line-numbers are from the code
  numberstyle=\tiny\color{mygray}, % the style that is used for the line-numbers
  rulecolor=\color{black},         % if not set, the frame-color may be changed on line-breaks within not-black text (e.g. comments (green here))
  showspaces=false,                % show spaces everywhere adding particular underscores; it overrides 'showstringspaces'
  showstringspaces=false,          % underline spaces within strings only
  showtabs=false,                  % show tabs within strings adding particular underscores
  stepnumber=2,                    % the step between two line-numbers. If it's 1, each line will be numbered
  stringstyle=\color{mymauve},     % string literal style
  tabsize=2,                       % sets default tabsize to 2 spaces
  title=\lstname                   % show the filename of files included with \lstinputlisting; also try caption instead of title
}
\usepackage{alltt}
\usepackage[sc]{mathpazo}
\usepackage{amsmath, amsthm, amssymb}
\usepackage{graphicx}
\usepackage[T1]{fontenc}
\usepackage{geometry}
\geometry{verbose,tmargin=2.5cm,bmargin=2.5cm,lmargin=1.5cm,rmargin=1.5cm}
\setcounter{secnumdepth}{2}
\setcounter{tocdepth}{2}
\usepackage{url}
\usepackage[unicode=true,pdfusetitle,
  bookmarks=true,bookmarksnumbered=true,bookmarksopen=true,bookmarksopenlevel=2,
breaklinks=false,pdfborder={0 0 1},backref=false,colorlinks=false]
{hyperref}
\hypersetup{pdfstartview={XYZ null null 1}}
\usepackage{float}
\usepackage{bm}
 %changes default sectioning commands -> 1,a, etc.
%\usepackage{breakurl}
\usepackage{lastpage}
\usepackage{fancyhdr}
\pagestyle{fancy}
%\usepackage[active,tightpage]{preview}
%\PreviewEnvironment{tikzpicture}
%%% Header and Footer %%% 
\lhead{}
\chead{\leftmark}
\rhead{}
\lfoot{Gonzales et al, cs351}
\cfoot{Final Project Specs}
\rfoot{Page \thepage\ of \pageref{LastPage}}
\IfFileExists{upquote.sty}{\usepackage{upquote}}{}


\begin{document}

\title{LameGame}
\author{Aaron Gonzales, Robert Nicholson, Weston Ortiz, Paige Romero, Hans
Weeks}
\maketitle
\section{LameGame (Final name undecided)
Specification}\label{lamegame-final-name-undecided-specification}


\section{Brief Description of
LameGame}\label{brief-description-of-lamegame}

LameGame is a first person shooter with spaceships. In LameGame, a
player will have a view of the spaceship from above and back, watching
it move around as it goes through space. The game environment will have
various obstacles, notably asteroids that can destroy a ship if not
avoided or destroyed by the user. In multiplayer mode, up to four
archrivals can work out their issues by attacking each other or tricking
a user into hitting an asteroid.

Littered throughout the game world, there will be various power-ups,
weaponry, and health refreshers that players may obtain by flying
through them. Some powerups may not be so powerful. Some may be very
powerful indeed.

\section{Scope}\label{scope}

The scope of LameGame is mostly the five developers, our TA, Torin, and
our instructor, Joel - unless for some reason the general public finds a
major interest in our amazing game, LameGame. It is meant to be
recreational for all except for the developers.

\section{Definitions}\label{definitions}

\begin{itemize}
\item
  Server: A machine that serves as a centralized point of entry for
  users during multiplayer access. It can store user data and gameplay
  information, calculate physics and interactions, and handle placement
  of the aformentioned all-powerful powerups. The server's physical
  location has not been determined, though a machine in FEC may suffice
  for our purposes. Users can connect to the server through the
  Internet, because it's 2014. Basic packet verification should be done
  to ensure only correct data is being sent to the server.
\item
  UI : The user interface for LameGame. This serves as the primary (and
  only) way of interacting with the program. Options to reload levels,
  start games, connect to the server, pull user data from the server,
  close the game, and so on will be present through the UI. The gui most
  likely will be done in OpenGL unless swing/java fx provide a better
  interface.
\item
  Player : Person who is playing the game to the best or worst of their
  abilities, in single-player or mulitplayer mode. Let no designation be
  made between `player' and `user'.
\item
  Input: as a subcomponent of the UI, input will be handled through a
  joystick (most likely wired xbox controller) and/or keyboard and
  mouse.
\item
  Graphics: LameGame will be highly visual and beautiful. Some (members
  of the dev team) may say it looks like Rothko or another famous artist
  who didn't work in concrete shapes or forms. LameGame's graphics will
  be made using Java OpenGL (JOGL) via the Lightweight Java Game Library
  (lwjgl). Developers will attempt to avoid creating new acronyms.
  Graphics will include meshes, lighting, textures, immersion, moveable
  cameras, and ``realistic'' movement of the objects in space, including
  the player's ship and all other things. This will be a large chunk of
  work.
\item
  Physics: We are looking to implement somewhat believable space physics
  with inertia playing a part in player movement/gameplay. We are in the
  process of looking at either implementing basics physics on our own or
  integrating bullet engine through JBullet. We also want to have
  asteroids moving through the game area and have the ability to
  explode, on contact with projectile or enemy ship.
\item
  World: The game's current world or map in which players may play. The
  world will be comprised of several decorative objects:

  \begin{itemize}
  \itemsep1pt\parskip0pt\parsep0pt
  \item
    the world will have a starry background that is everpresent, like
    the stars themselves.
  \item
    The world will have far-away planets to provide a sense of place.
    These planets will hopefully move a little bit, because everything
    is moving in space.
  \item
    the world will have 3d objects in it, primarily asteroids unless we
    have time to add more objects (e.g., space mines, space pirates,
    space ninjas, space princesses, space beach balls, space aliens,
    floating software specifications that devour you in one `hit', evil
    Computer Science instructors that think you have nothing else to do
    but work on their class).
  \end{itemize}
\item
  Testing: A thing that young developers seldom do correctly but is
  crucial to the success of a larger project. Testing within LameGame
  will include three subcomponents:

  \begin{itemize}
  \itemsep1pt\parskip0pt\parsep0pt
  \item
    Unit tests - these shall be written for each class or collection of
    classes that inherit each other.
  \item
    Integration test - these shall be performed to ensure that the
    disparate parts are hacked together in a way that works and doesn't
    break some other component in the process. It is the author's
    suspicion that this will be challenging.
  \item
    `Daily' builds - after the first semi-working version is complete,
    one member will be in charge of creating a `daily build' of
    LameGame. This daily build will be a final bugfinding and
    integration testing measure . This daily build may not in fact be
    daily, but may happen many times per day as the deadline approaches.
    Also, the authors do not expect that compiling LameGame will take
    that long - it's not Windows Vista, after all.
  \end{itemize}
\item
  Version control / code reviews: Git through Github. We students will
  be using Git/Github as a DVCS, implementing a standard
  feature-and-bugfix branch workflow because we all want to do things
  that future employers may like. A `dev' branch and a `master' branch
  will be the primary lines of development, with merges into master
  being handeld by one crazy group member. We will make extensive use of
  feature branches and pull requests through github to ensure that code
  standards, code reviews, and tests are implemented correctly.
\item
  Game interaction: Any interaction between a Player and the world. This
  includes events such as picking up an incredible powerup or firing a
  space plasma rocket grenade laser at an archnemesis.
\item
  Documentation: Refers to both JavaDoc, specifications documents like
  this one, notes, and smoke signals that relay information to each
  other or to an end user/player. This includes technical and
  non-technical docs.
\end{itemize}

\section{General breakdown of
responsibilities}\label{general-breakdown-of-responsibilities}

Note that distribution of labor may change slightly as the project's
completion waxes and wanes, but for right now, our breakdown is as
follows:

\subsection{Aaron Gonzales}\label{aaron-gonzales}

\begin{itemize}
\itemsep1pt\parskip0pt\parsep0pt
\item
  ``Leader'' or something like it.
\item
  Tracking development process and assigning release dates
\item
  Testing
\item
  Daily builds
\item
  VC/ Code reviews
\item
  Assist with server and physics implementations
\item
  OO design and layout for subcomponents
\item
  Performance testing
\item
  Documentation
\item
  Pedantic musings
\end{itemize}

\subsection{Robert (Dan) Nicholson}\label{robert-dan-nicholson}

\begin{itemize}
\itemsep1pt\parskip0pt\parsep0pt
\item
  Implementing graphics (not as simple as it may sound!)
\item
  Implementing Audio / OpenAL
\item
  Moral support
\end{itemize}

\subsection{Weston Ortiz}\label{weston-ortiz}

\begin{itemize}
\itemsep1pt\parskip0pt\parsep0pt
\item
  General physics and game interaction/logic
\item
  Helping with the server and communication
\item
  Collisions / network related logic for keeping collisions correct on
  client side, while server does most of the collision checking
  (hits/pickups)
\item
  `2nd lead' assisting with integration if needed
\end{itemize}

\subsection{Paige Romero}\label{paige-romero}

\begin{itemize}
\itemsep1pt\parskip0pt\parsep0pt
\item
  Graphics implementation
\item
  Models, textures, lighting
\item
  UI/UX for game
\end{itemize}

\subsection{Hans Weeks}\label{hans-weeks}

\begin{itemize}
\itemsep1pt\parskip0pt\parsep0pt
\item
  Server and client communication
\item
  Containers for client game-state update
\item
  Game physics
\end{itemize}



\end{document}
